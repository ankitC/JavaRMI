\documentclass[11pt,letterpaper]{article}

\usepackage{amsmath}
\usepackage{amssymb}
\usepackage{amsthm}
\usepackage{graphicx}
\usepackage{mathtools}
\usepackage{enumerate}

\oddsidemargin0cm
\topmargin-2cm
\textwidth16.5cm
\textheight23.5cm

\usepackage[cm]{fullpage}

\setlength{\jot}{8pt}

\newcounter{qnum}
\setcounter{qnum}{1}

\DeclarePairedDelimiter{\ceil}{\lceil}{\rceil}
\DeclarePairedDelimiter{\floor}{\lfloor}{\rfloor}

\newcommand{\q}{\vspace{.25in} \hrule\vspace{0.5em}
\noindent{\bf Problem \arabic{qnum}}\addtocounter{qnum}{1} \vspace{0.5em}
\hrule \vspace{.10in}}

\newcommand{\h}[1]
                    {\vspace{.25in} \hrule\vspace{0.5em}
                    \noindent{\bf #1 \vspace{0.5em}}
                    \hrule \vspace{.10in}
                    }

\renewcommand{\part}[1] {\vspace{.10in} {\bf (#1)}}

%\newcommand{\qed}{\hfill \mbox{\raggedright \rule{.07in}{.1in}}}

\newcommand{\myname}{Vinay Balaji}
\newcommand{\partnername}{Ankit Chheda}
\newcommand{\myandrew}{vbalaji@andrew.cmu.edu}
\newcommand{\partnerandrew}{achheda@andrew.cmu.edu}
\newcommand{\myhwnum}{Project \#2}

\setlength{\parindent}{0pt}
\setlength{\parskip}{5pt plus 1pt}

%\input{defs.tex}

\begin{document}

\medskip

\thispagestyle{plain}
\begin{center}                  % Center the following lines
{\Large 15-440 \myhwnum} \\
10 October 2013 \\
\myname\hfill\myandrew \\
\partnername\hfill\partnerandrew \\
\end{center}

\h{Design}

The design we chose for this lab is very similar to Java's original RMI design.\\
\\
At least three actors are needed: a client, a server, and a registry.\\
\\
To begin, the registry listens for incoming LOOKUP, LIST, BIND, and REBIND requests, all of which are wrapped in the RegistryRequestMessage object.\\
\\
The requests are processed as follows:
\begin{enumerate}[i]
\item The LIST command returns a list of all the registered objects that exist with that registry.
\item The LOOKUP command returns the remoteObjectReference of the requested object.
\item The BIND command binds the given remote object reference to the specified unused ID in the registry, allowing for future lookups.
\item The REBIND command binds the given remote object reference to the specified ID, overwriting a previous reference if it exists.
\end{enumerate}

All messages sent to the registry are encapsulated in request-message objects.\\
Similarly, all messages sent from the registry are encapsulated in return-message objects.\\
\\
The server(s) registers its remote objects by going through a registry messenger, which tells the registry to register a remote object with the specified ID and stub (interface). Server methods invoked remotely can return other remote object references by either already having the remote reference in the class containing said method, or can look up the appropriate reference in the registry.  \\
\\
The server(s) uses a helper to listen to RMI messages and invoke the methods on the appropriate local (server) objects.\\
\\
The client(s) does not query the registry directly either, but rather goes through the same messenger used by the server. Through the messenger, the client queries the registry for a list of remote object references available, and then requests one or more of them. The remote object reference returned has a method to return its stub, from which we can finally make remote method invocations.\\
\\
The stubs construct an invocation message to send off to the server helper, which is actually the execution skeleton / dispatcher. This execution skeleton invokes the method and packages the return value (or exception) into a message back to the stub. The stub then finally extracts the return value from the message and returns this value if it is not an exception; if the value is an exception, it is thrown.

\newpage

\h{Major Design Decisions}

Our framework has both clients and servers using the same message object, the RegistryRequestMessage, to communicate with the Registry. This allows for less distinction between the roles of a server and of a client. For example, two machines could both start as "servers" and register their own remote objects, but later these machines become "clients" by invoking the remote methods of the other machine.\\
\\
For similar reasons, we encapsulate all messages from the Registry within a ReturnMessage object, which communicates whether a request to the registry was properly fulfilled or not. This message can contain either the return value, or an exception if one occured.\\
\\
In order to more clearly separate receiving RMI requests from actually handling them, the server helper (which represents the skeleton) has two components:
\begin{enumerate}[1)]
\item The helper itself only listens for incoming RMI requests, and spawns a new thread for each request.
\item The execution of the request occurs in an ExecutionAgent, which is also responsible for communicating the results back to the client. Each agent runs on its own thread, which was created by the helper.
\end{enumerate}

The stubs are implemented as Proxy objects with our own custom RemoteInvocationHandler. This handler marshalls all method invocations into a bean called a RemoteInvocationMessage, which is then sent to the server helper / execution skeleton. 

\h{How to Run}

Our example involves querying "databases" for the capitals of nations and states.\\
These databases are stored in an archive, from the client requests specific databases.\\
\\
The archive and all of its databases have remote object references on the registry.\\
\\
The references to the databases can either be retrieved from the registry directly, or by a method call on the archive, which returns the appropriate remote object reference.\\
\begin{enumerate}[1)]
\item The first step towards getting our project up and running is to make some changes to the properties.txt file located in the project root.\\
\\
Here, the parameters of interest are:
\begin{enumerate}[i]
\item serverPort: If starting a server instance, this port will be used
\item clientPort: If starting a client instance, this port will be used
\item registryPort: If starting a registry instance, this port will be used
\item registryIPAddress: Set to IP of the machine that the registry will run on.\\
\hspace*{93pt}Used by clients and servers to connect to registry, ensure that this is correct!
\end{enumerate}

\newpage

\item The next step is to run one or more of the following scripts:
\begin{enumerate}[i]
\item package.sh: compiles all java files and creates a jar
\item execute.sh $<$Registry / Server / Client$>$: starts the specified instance
\item clean.sh: deletes all generated class files and the jar file
\item run.sh $<$Registry / Server / Client$>$:	\\
same as running: package.sh,  
		followed by execute.sh $<$Registry / Server / Client$>$
\end{enumerate}

\item For our example, please start the instances in the following order:
\begin{enumerate}
\item Registry
\item Server
\item Client
\end{enumerate}
	

\item The output should demonstrate our framework's capability to:
\begin{enumerate}[i]
\item Register (bind) remote object references on the registry
\item List remote object references currently on the registry
\item Look up remote object references from registry
\item Get local stubs for remote object references
\item Invoke method on stub, which can return, a value, an exception,
	   or another remote object reference.
\item Propagate exceptions thrown from remote procedure call
\end{enumerate}

\end{enumerate}

\h{Assumptions}

\begin{enumerate}
\item The following file(s) are located at the execution directory of EVERY instance:
\begin{enumerate}
\item properties.txt
\end{enumerate}

\item The following file(s) are located at the execution directory of each SERVER instance:
\begin{enumerate}
\item nations.txt
\item states.txt
\end{enumerate}
	
\item The ports specified in the configuration file are not being used by other processes.

\end{enumerate}

\end{document}